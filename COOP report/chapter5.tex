\chapter{สรุปผล}
\label{chapter:conclusion}

ตลอดระยะเวลาการที่ปฏิบัติงานสหกิจศึกษา ณ บริษัท ไซเจ็น จำกัด นักศึกษาได้รับหน้าที่ให้พัฒนา เว็บแอพพลิเคชั่นประเมินความสามารถเบื้องต้นของผู้สมัครงาน โดยเว็ปแอพพลิเคชั่นนี้ ได้ถูกพัฒนาด้วยการเน้นส่วนที่จัดการกับฐานข้อมูล(back-end) เป็นหลัก โดยในส่วนนี้นักศึกษาสามารถทำสำเร็จลุล่วงไปด้วยดี สามารถใช้งานได้ทุกฟังก์ชันตามที่ได้ทำการออกแบบไว้ และส่วนต่อประสานกับผู้ใช้(front-end) ณ ตอนนี้ เป็นการออกแบบมาเพื่อให้ระบบใช้งานได้โดยที่ไม่เน้นเรื่องความสวยงามเป็นหลัก เนื่องจากมีระยะเวลาที่จำกัด ในส่วนนี้จึงได้ส่งมอบให้พี่ในทีม ได้นำเว็บแอพพลิเคชั่นไปพัฒนาส่วนต่อประสานกับผู้ใช้ต่อไป เพื่อให้มีการใช้งานที่สะดวกและสวยงามมากยิ่งขึ้น จากการที่ได้พัฒนาระบบ ทำให้นักศึกษาได้เรียนรู้ กระบวนการทำงานในชีวิตจริง ได้รู้วิธีการจัดสรรเวลาและวางแผนการทำงานต่างๆให้มีประสิทธิภาพ รวมถึงทักษะในการสื่อสาร เพราะเป็นการทำงานร่วมกับผู้คนหลายช่วยวัย สุดท้ายนี้นักศึกษาได้เห็นว่าการเข้าร่วมสหกิจศึกษา ณ บริษัท ไซเจ็น จำกัด ทำให้นักศึกษาได้รับประสบการณ์ในการทำงาน ที่ไม่สามารถหาได้จากห้องเรียน ซึ่งนักศึกษาจะนำความรู้ความสามารถที่ได้รับนี้ไปปรับใช้เพื่อให้เกิดประโยชย์ต่อตัวนักศึกษาเอง และผู้อื่นด้วยเช่นกัน

\section{ประโยชน์ที่ได้รับจากการปฏิบัติงาน}

\subsection{ประโยชน์ต่อตนเอง}

จากการที่ได้เข้าร่วมสหกิจศึกษาเป็นระยะเวลา 4 เดือน ทำให้นักศึกษาได้รับความรู้ความสามารถ เสมือนการเรียนรู้นอกห้องเรียน

\subsection{ประโยชน์ต่อสถานประกอบการ}

\subsection{ประโยชน์ต่อมหาวิทยาลัย}

\section{วิเคราะห์จุดเด่น จุดด้อย โอกาส อุปสรรค(Swot Analysis)}

\subsection{จุดเด่น}

\subsection{จุดเด้อย}

\subsection{โอกาส}

\subsection{อุปสรรค}

\section{ประสบการณ์ที่ประทับใจ}